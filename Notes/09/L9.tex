\documentclass[12pt]{article}

\pagenumbering{gobble}
\usepackage{mypython, mymath}

\usepackage[margin=0.6in]{geometry}

\begin{document}
\begin{center}
   \LARGE Lecture 9
\end{center}

\section*{Lists}

\lstinputlisting{list_stuff.py}

\noindent
This means that we can use \lstinline{assertEqual}!  \textbf{But}, not
\lstinline{assertAlmostEqual}\dots, \emph{so} for Lab 5, an
\lstinline{assertListAlmostEqual} is provided!

\lstinputlisting{more_lists.py}

\section{List Reference Passing}

However, lists are really odd/counter-intuitive in some ways when dealing with
functions.

\lstinputlisting{scope.py}

\section*{Tuples}

\lstinputlisting{tuple_stuff.py}

\section{Maps}

Let's write a function dealing with lists!  Given a list, write a function that
computes (and returns) a new list with double the entries in the original list.
(Do first with while loop, then for, then list comprehension.  Mention that
list comprehensions are inspired by mathematical notation.  Don't forget to
write tests!)  We call this kind of function a map.  It's mapping each value in
the old list to a corresponding value in the new list via some kind of
function.  We're ``mapping'' the doubling function onto the list.

\section{Filters}

Let's write more functions dealing with lists!  Given a list, write a function
that computes (and returns) a new list with only the even values from the
original list.  (Do first with while loop, then for, then list comprehension.
Don't forget to write tests!)  We call this kind of function a filter.  It's
filtering out all the undesired elements.

\end{document}
