\documentclass[12pt]{article}

\pagenumbering{gobble}
\usepackage{mypython, mymath}

\usepackage[margin=0.6in]{geometry}

\begin{document}
\begin{center}
   \LARGE Loops!
\end{center}

\begin{enumerate}
   \item Write a snippet of code to prompt the user for a number.  ``Count'' up
      to the input number starting from the number 1.  Sample output is shown
      below (user input in bold).  Use a Python while-loop to do the number
      printing.

      \begin{lstlisting}[language=]
   Enter a number: #/*\textbf{6}*/
   1
   2
   3
   4
   5
   6
      \end{lstlisting}

      \vfill

   \item Write a snippet of code to prompt the user for a number.  ``Count''
      down to zero from the input number.  Sample output is shown below (user
      input in bold).  Use a Python while-loop to do the number printing.

      \begin{lstlisting}[language=]
   Enter a number: #/*\textbf{6}*/
   6
   5
   4
   3
   2
   1
   0
      \end{lstlisting}

      \vfill
      \newpage

   \item Write a snippet of code to keep track of canned food drive donations.
      Using a Python while-loop, prompt the user for how many cans they will
      donate.  Keep asking for donations until you receive at least 100 cans.
      Then print out how many ``people'' donated and the total cans collected.
      A sample run is below (user input in bold):

      \begin{lstlisting}[language=]
   How many cans? #/*\textbf{12}*/
   How many cans? #/*\textbf{35}*/
   How many cans? #/*\textbf{3}*/
   How many cans? #/*\textbf{15}*/
   How many cans? #/*\textbf{7}*/
   How many cans? #/*\textbf{21}*/
   How many cans? #/*\textbf{9}*/
   Total donations: 7
   Total number of cans: 102
      \end{lstlisting}

      \vfill

   \item Write a Python while loop that says hello to the user.  Then ask the
      user if they would like another hello.  If they type a `y' for yes, say
      hello again and then ask them if they would like another hello.  If they
      type anything else, say goodbye and quit the program.  A sample run is
      below (user input in bold):

      \begin{lstlisting}[language=]
   Hello!
   Again (y/n)? #/*\textbf{y}*/
   Hello!
   Again (y/n)? #/*\textbf{y}*/
   Hello!
   Again (y/n)? #/*\textbf{n}*/
   Goodbye!
      \end{lstlisting}

      \vfill
      \newpage

   \item Write a Python while-loop that displays the squares of all the numbers
      from 0 to a given number (for example all the integers between 0 and 3
      inclusive). The end of the range will be specified in an integer named
      ``last''. Assume this variable has been declared and initialized earlier
      in the program to the desired values (for example: last = 3;)
      Additionally, keep track of the sum of all the squares and display the
      sum after displaying the squares. Declare any additional variables that
      you need.  Make sure that the output matches the example lines below:

      \begin{lstlisting}[language=]
   The number 0 squared is 0.
   The number 1 squared is 1.
   The number 2 squared is 4.
   The number 3 squared is 9.
   The sum of the squares is 14.
      \end{lstlisting}
\end{enumerate}


\end{document}
