\documentclass[12pt]{article}

\pagenumbering{gobble}
\usepackage{mypython, mymath}

\usepackage[margin=0.6in]{geometry}

\begin{document}
\begin{center}
   \LARGE Lecture 5
\end{center}

\setcounter{section}{-1}
\section{Queries}

\section{More Loops}

Let's write more loops!  In particular, let's write a loop that will continue
until the user wants it to stop.  (Ask them how to do it!)

\lstinputlisting{more_loops.py}

\noindent
Now let's write a function that takes a number as input.  It then asks the user
for values until they total more than that number.  Then we'll print the total.

\lstinputlisting{more_loops2.py}

\section{Loops Practice}

Give loops worksheet!  Go over worksheet!  For 5, spend some time with
padding options.

\section{Keyword Arguments (for \lstinline{print})}

There have been many questions about how to print without the line break at the
end.  Let's talk about that!  By default, Python always goes onto the next line
after a print.  But we can tell Python not to print the newline character at
the end.

\lstinputlisting{printing.py}

\noindent
Here I'm telling Python to print nothing (an empty string) after it prints what
I told it to print.  If I set \lstinline{end} to be a newline character (as
seen in Lab 1), I would get the same behaviour as before.

\section{Nested Loops}

We can have loops inside other loops!  Let's look at a few and see if we can
figure out what they do.

\section*{Lazy Import}

There's another way of importing.

\begin{lstlisting}
from math import sqrt
\end{lstlisting}

\noindent
Upside: less typing, downside: pollute namespace

\section*{Project 2}

Go over spec, show them what they get, show them executable.  Talk about I/O
testing.

\end{document}
