\documentclass[12pt]{article}

\pagenumbering{gobble}
\usepackage{mypython, mymath}

\usepackage[margin=0.6in]{geometry}

\begin{document}
\begin{center}
   \LARGE Lecture 14
\end{center}

\section*{Queries}

\section{Printing Objects}

Some of you may have noticed that when you print an object (an instance of a
class) like a point, Python does something weird/not useful.  Python is telling
us the location of the Point in computer memory.  This could be useful
sometimes but often it isn't very.  Let's change it!

\lstinputlisting{pet.py}

\noindent
Play around with \lstinline{str}/\lstinline{repr} method.  Show that
\lstinline{str} is called by print and \lstinline{repr} is called on command
line.

\section{Dictionaries}

Uncomment one section at a time

\lstinputlisting{dictionary.py}

\section{Exceptions}

What happens when I do the following?

\lstinputlisting{get_num.py}

\noindent
To ask a slightly different question, what could go wrong?  What if I type in
``five''?  What happens then?

It crashes!  More specifically, it raises an exception!  More specifically, it
raises a \lstinline{ValueError} (this will be important in a moment).  But what
if I don't want the program to crash just because of a slight error?  I can
tell python to `try' to do something and do something else of an exception is
raised.

\lstinputlisting{get_num2.py}

\noindent
Let's modify the code so that I keep asking for a value until it's an integer.

\lstinputlisting{get_num3.py}

\noindent
Here I'm saying to `try' to do a possibly unsafe operation, and if a
\lstinline{ValueError} occurs, instead do something else.

\end{document}
