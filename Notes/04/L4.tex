\documentclass[12pt]{article}

\pagenumbering{gobble}
\usepackage{mypython, mymath}

\usepackage[margin=0.6in]{geometry}

\begin{document}
\begin{center}
   \LARGE Lecture 4
\end{center}

\setcounter{section}{-1}
\section{Queries}

\section{Main Motivation and TDD}

Make a file \lstinline{funcs.py}.  Make two functions:

\lstinputlisting[frame=single]{funcs.py}

\noindent
Now I can write and run the tests without having actually written the
functions!  Let's write some tests!

\lstinputlisting[frame=single]{funcs_tests.py}

\noindent
At this point all the tests will fail.  That's fine!  They should!  We never
wrote them!  But now we can write the functions little by little and we'll know
we're done when all the tests pass!  So now we write \lstinline{f}.  (Do this.)
Before even running the tests, we may want to verify that the function kind of
does what we expect, a sanity check of sorts.  So let's add a \lstinline{print}
call.  Okay, it looks like its working.  Let's run the tests!
\vspace{2ex}


\noindent
Uh oh!  There's some garbage being printed in my testing output!  That's our
print!  Let's put the prints in \lstinline{main}!

\section*{Mutation}

Variables can change values!  Think of \lstinline{x} as a box that can only
contain one value.  (Draw this!)

\begin{lstlisting}
>>> x = 5
>>> print(x)
5
>>> x = 10
>>> print(x)
10
>>> x = x + 1
>>> print(x)
11
>>> x += 1
>>> print(x)
12
>>> x -= 5
>>> print(x)
7
\end{lstlisting}

\section{Scope}

Draw out the variables in different scopes in \lstinline{scope.py}.  Also
probably mention the decimal format specifier.

\section{Tracing Code}

Go through \lstinline{weird.py} one line at a time noting the variables at each
line.  Put variables in parentheses if they go out of scope (and remove them
entirely when they no longer exist).  (Have code/tracing paper printed out!)

\begin{lstlisting}
x: 8.5
y: hello
\end{lstlisting}

\begin{lstlisting}
x: 5
y: hi
\end{lstlisting}

\section{Loops}

What if I want to write code that repeats?  Let's say I want to print out the
numbers from $10 \searrow 1$.  I could do something like:

\begin{lstlisting}
print(10)
print(9)
#/*\vdots*/
print(1)
\end{lstlisting}

\noindent
But\dots that's a lot of typing.  I don't want to do that\dots  Instead, we can
use loops!

\begin{lstlisting}
counter = 10
while counter > 0:
   print(counter)
   # What would go wrong if I stopped here?
   counter -= 1
\end{lstlisting}
\end{document}
