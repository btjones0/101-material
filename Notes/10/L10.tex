\documentclass[12pt]{article}

\pagenumbering{gobble}
\usepackage{mypython, mymath}

\usepackage[margin=0.6in]{geometry}

\begin{document}
\begin{center}
   \LARGE Lecture 10
\end{center}

\section*{Queries}

Mention that the last one is a fold pattern!  They'll see more later/also in
Lab 6.

\section{List of Tuples}

Go through the example (list comprehension mapping $x \rightarrow (x, x^2)$ in
a list).  Also print specific tuples and then specific values from the tuples.

\section{Project 3}

Talk about list of lists and Project 3.  Should start \emph{NOW}.  Read through
the specification carefully.  Go over description and then walk through big
example \lstinline[language=]{p1.in}.  Quit after a bit and then run program to
solve it.  It worked!  Now work though entire little example.

\section{Lists of Lists}

Write a 2 dimensional sum function.

\section{Strings}

How exactly are strings stored in a computer?

\subsection{Characters}

\lstinputlisting{chars.py}

\subsection{Character Functions}

Let's write a function called \lstinline{char_upper} that will take a single
character and return the uppercase version.

\subsection{Strings}

\lstinputlisting{strings.py}

\subsection{String Functions}

Let's write a function called \lstinline{str_upper} using
\lstinline{char_upper}.

\section{Searching}

Write a \lstinline{contains} function and an \lstinline{index_of} function.

\end{document}
