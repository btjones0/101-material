\documentclass[12pt]{article}

\pagenumbering{gobble}
\usepackage{mypython, mymath}

\usepackage[margin=0.6in]{geometry}

\begin{document}
\begin{center}
   \LARGE Lecture 6
\end{center}

\setcounter{section}{-1}
\section{Queries}

\section*{Lazy Import (in 1st section)}

There's another way of imported.

\begin{lstlisting}
from math import sqrt
\end{lstlisting}

\noindent
Upside: less typing, downside: pollute namespace

\section{For Loops}

For loops are another kind of Python loop.  They are great if we know in
advance how many times we want to do things.

\begin{lstlisting}
for count in range(10):
   print(count)
\end{lstlisting}

\noindent
Let's revisit problems 1 and 2 from the loops worksheet.

\begin{lstlisting}
num = int(input("Enter a number: ")) # Number 1
for x in range(1, num + 1):
   print(x)

num = int(input("Enter a number: ")) # Number 2
for x in range(num, -1, -1):
   print(x)
\end{lstlisting}

\noindent
Numbers 3 and 4 are not a good fit for a for loop; why?  But number 5 is!  Try
it on your own!  Come show me in office hours!

\section{Nested Loops}

We can have loops inside other loops!  Let's look at a few and see if we can
figure out what they do.

\section*{String Operations}

We can add strings!  We can multiply strings and ints!

\section{Default Values}

Let's write a distance function!  (Then modify it to default to distance from
origin!)

\lstinputlisting[frame=single]{funcs.py}

\lstinputlisting[frame=single]{funcs_tests.py}

\section*{Lists}

\begin{lstlisting}
my_list = [2, 3, 7.1, "hi"]
my_list
my_list[0]
my_list[3]
my_list[4]
my_list[100] # Guesses???
my_list[-2] # Guesses???

my_list[2] = -45.1 # Lists are mutable!!!
my_list
my_list.append(9) # we can add more stuff to the end!

list2 = [8, 0]
my_list + list2
my_list
list2
my_list = my_list + list2
my_list
my_list * 2
my_list
my_list *= 2
\end{lstlisting}

\end{document}
