\documentclass[12pt]{article}

\pagenumbering{gobble}
\usepackage{mypython, mymath}

\usepackage[margin=0.6in]{geometry}

\begin{document}
\begin{center}
   \LARGE Lecture 1
\end{center}
\section{Basic Unix Commands}

That is, how to use the black box called terminal.

\begin{lstlisting}[language=bash]
ls
pwd
ls -a # (come back to . and ..)
ls -l
ls -al
mkdir CPE101
ls # (with flags?)
cd CPE101
ls
pwd
mkdir Lecture01
cd Lecture01
ls
pwd
cd .. # (oh!  magic!, explain ..)
cd ~
cd /Users/btjones/CPE101/Lecture01
cp ~/Files/smallfile . # (explain .)
ls -l
cat smallfile
cp ~/Files/largefile .
ls -l
cat largefile
less largefile # (q to quit)
rm smallfile # (be VERY careful with this!)
cp ~/Files/smallfile .
mv smallfile ..
cd ..
ls
mv smallfile ..
cd ..
ls
mv smallfile Lecture01
cd Lecture01
mv smallfile newfile # (can use mv to rename)
man mv # (forget how to use mv?  q to quit)
\end{lstlisting}

\section{Python}

We're using Python 3 vs 2, there are some minor differences that we won't talk
about.

\noindent
Run interpretor, we can use it as a calculator!

\begin{lstlisting}
2 + 2
3 * 4
10 - 26
7 / 2
7 // 2
x = 5 # variables!  math!
x * 3
y = 2 * x + 4
y
greeting = "Hello" # we can store text!  we call these strings!
x > 2
y < 4
\end{lstlisting}

\noindent
Variables, assigning values to them, show as a box.  Use variable names that
mean something!  Some words cannot be used.

\section{Operations}

Python has a bunch of operators (+, -, *, /, \%, //, **).

\noindent
What is the value of \lstinline{3 + 2 * 4}

\noindent
What about \lstinline{3 + 2      * 4}?

\noindent
\lstinline{(3 + 2)*4}?

\subsection{Order of Operation}

\begin{itemize}
   \item Parentheses
   \item \lstinline{**} (right -> left)
   \item unary \lstinline{-} (like \lstinline{-5})
   \item *, /, //, \% (left -> right)
   \item +, - (left -> right)
\end{itemize}

\subsection{Evaluating Expressions}

Have them guess:
\begin{itemize}
   \item \lstinline{-3**2+7.0//2 # = -6.0}
   \item \lstinline{3 + 6 // 4 - 2 ** 2 + 8 / 2.0 / (2 + 1) - 25 % 4 # = 0.3333333333}
   \item \lstinline{8 / 3 # = 2.6666666666666665}
\end{itemize}

\section{Editing Files}

We can also put code in a file!  Python files end with the extension
\lstinline{.py}.

\begin{lstlisting}[language=bash]
cp ~/Files/code.py .
cat code.py
python3 code.py
\end{lstlisting}

\section{Testing}

\begin{lstlisting}
cp ~/Files/testing.py .
\end{lstlisting}

\noindent
Talk about unit testing!  Talk about floats sucking!

\end{document}
