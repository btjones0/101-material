\documentclass[12pt]{article}

\pagenumbering{gobble}
\usepackage{mypython, mymath}

\usepackage[margin=0.6in]{geometry}

\begin{document}
\begin{center}
   \LARGE Lecture 15
\end{center}

\section*{Queries}

\section{Command line Arguments}

We've seen lots of commands (ls, mkdir, cd, cp, cat, less, rm, mv) that take
input via the command line; they don't make you run them and then wait for
input.  Let's see how to do that!

\lstinputlisting{cmdline.py}

\noindent
Great!  Let's use our new found knowledge to change an earlier example.

\section{I/O Example with Command Line Args and Error Handling}

Let's write a piece of code that takes the name of a file as a command line
argument.  It will then print out the line along with its line number.

\lstinputlisting{files1.py}

\noindent
This will work great\dots as long as the user gives me the name of a file
(otherwise \lstinline{IndexError}).  And not just any file, it also has to
exist (otherwise \lstinline{FileNotFoundError}), and I have to have the
permission to read it (otherwise \lstinline{PermissionError}).  So\dots let's
try again and handle some errors!

\lstinputlisting{files2.py}

\noindent
Now, if an exception happens, we handle it\dots but we have no idea \emph{what}
went wrong.  We can tell Python to handle each type of exception differently!

\lstinputlisting{files3.py}

\section{Finally}

In addition to what we've already seen, there is another thing we can do with
exception handling, the \lstinline{finally} clause.


\lstinputlisting{fin.py}

\noindent
Code in the \lstinline{finally} will always happen, whether or not an exception
was raised.  It will also run whether or not we handled the exception.
(\lstinline{raise} various errors to demonstrate.)

\section{Sorting}

If time, talk about insertion and/or selection sort.  Ask the audience how to
sort a list and go with it?
\[[25,\ 13,\ -100,\ -12,\ 60,\ 31,\ -76,\ 2]\]

\subsection{Insertion Sort}

\begin{itemize}
   \item Start at index 1, swap with lower one until it gets to the correct
      location (insert it).
   \item Each iteration, the first \(i\) elements of the list are sorted, each
      subsequent element is then inserted.
\end{itemize}

Go through example above.  Write out code?

\lstinputlisting{ins_sort.py}

\subsection{Selection Sort}

\begin{itemize}
   \item Find the minimum (select it), swap with index 0
   \item Repeat for spots \(1,\ 2,\ \dots,\ n-1\)
\end{itemize}

Go through example above.  Write out code?

\lstinputlisting{sel_sort.py}

\end{document}
