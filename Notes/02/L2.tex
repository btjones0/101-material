\documentclass[12pt]{article}

\pagenumbering{gobble}
\usepackage{mypython, mymath}

\usepackage[margin=0.6in]{geometry}

\begin{document}
\begin{center}
   \LARGE Lecture 2
\end{center}

\setcounter{section}{-1}
\section{Queries!}

Talk about comments.  For query 4, explain multiple tests!

\section{Expressions}

\subsection{Casting}

\begin{lstlisting}
x = 3.8
int(x)
x
\end{lstlisting}

\subsection{Evaluate}

Show them and then vote!  \textbf{Note: floats have at least one decimal place
of precision, do this on your midterms!}

\section{Multiple Files}

So far everythin we've done has been in one file.  This is \emph{very} unlikely
for large scale projects.  Programs could have 100s of files.  In Python, each
file is called a module.

\noindent
Show \lstinline{my_time.py}.  What if I want to use these in a different file?

\noindent
Make \lstinline{use_time.py}

\begin{lstlisting}
import my_time

print("The number of seconds in a day is", seconds_per_day)
\end{lstlisting}

\noindent
This won't work!  I didn't say where \lstinline{seconds_per_day} came from!
Add module name and \lstinline{.}

\section{Functions}

\subsection{Built-ins}
Hopefully you've already seen the idea of functions in a math class at some
point.  Let's take a look at how to use and then define our own functions.
Python already has a bunch of functions built in.

\begin{lstlisting}
len("Monday")
abs(-10)
int(32.9)
float(1)
sqrt(4) # doesn't exist
# But!  There is a sqrt function in the math module
import math
math.sqrt(4)
# The math module has a lot more stuff
# some of which we will see later
\end{lstlisting}

\subsection{Defining Functions}

We can also define our own functions!  Why should we?  Write code once, test
it, reuse it.  Don't repeat code, put it in a function.  In general, a function
should do \emph{one} thing and it should do that one thing really well.

\noindent
Let's make \(f(x) = 3x + 2\) in Python!  (Put in \lstinline{funcs.py})
\textbf{\emph{TALK ABOUT INDENTATION!!}}

\begin{lstlisting}
def f(x):
   return 3*x + 2
\end{lstlisting}

\noindent
Python functions can be a little more general than mathematical ones.  Write
\lstinline{is_zero} function.  Again, variable names can be verbose.  Introduce
\lstinline{==} (and \lstinline{!=} I guess?)

\begin{lstlisting}
def is_zero(value):
   return value == 0
\end{lstlisting}

\noindent
Write \lstinline{area_of_rectangle(x1, y1, x2, y2)}.

\hfill
\begin{tikzpicture}
   \draw (0, 3) node[anchor=north east] {($x_1$, $y_1$)}
   rectangle (4, 0) node[anchor=south west] {($x_2$, $y_2$)};
\end{tikzpicture}

\begin{lstlisting}
def area_of_rectangle(x1, y1, x2, y2):
   width = abs(x2 - x1)
   height = abs(y1 - y2)
   return width * height
\end{lstlisting}

\noindent
Then make a \lstinline{use_funcs.py} and \lstinline{test_funcs.py}


\end{document}
